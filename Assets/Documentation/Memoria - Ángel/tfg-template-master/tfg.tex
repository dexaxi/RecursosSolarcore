% Clase del documento
\documentclass[12pt,twoside,titlepage]{report}





%%%%%%%%%%%%%%%%%%%%%%% Paquetes %%%%%%%%%%%%%%%%%%%%%%%

\usepackage[a4paper,bindingoffset=3mm,bottom=35mm]{geometry}


% Usad \usepackage[dvips]{graphicx} o \usepackage[pdftex]{graphicx} (no ambos)
%\usepackage[dvips]{graphicx} %%% para LaTeX. Las figuras deben estar en formato eps

\usepackage[colorlinks=true,pdftex]{hyperref}   %%% Opcional. Para incluir marcadores y enlaces en el pdf
\usepackage[pdftex]{graphicx}  %%% para pdflatex. Las figuras pueden estar en pdf, jpg, svg y otros formatos


\usepackage[spanish]{babel}

%\usepackage[latin1]{inputenc} % Usad en WinEdt/MikTex
\usepackage[utf8]{inputenc} % Usad en overleaf

%\usepackage[T1]{fontenc}


% Algunos paquetes útiles

\usepackage{amsmath,amssymb}
\usepackage{hyperref}
\usepackage{xcolor}
\usepackage{afterpage}
\usepackage{paralist}
\usepackage{array}
\usepackage{enumerate}
\usepackage{paralist}
\usepackage{enumitem}
\usepackage{float}
\usepackage{setspace}
\usepackage{listings}
\usepackage{algorithm}
\usepackage{algorithmic}
\usepackage{fancyhdr}
\usepackage{rotating}
\usepackage{multirow}
\usepackage{animate}
\usepackage{movie15}
\usepackage{media9}
\usepackage{graphicx}
\usepackage{epstopdf}

% Otros paquetes

\usepackage{quotchap}
\usepackage{lipsum}

%%%%%%%%%%%%%%%%%%%%%%%%%%%%%%%%%%%%%%%%%%%%%%%%%%%%%%%%

%%%%%%%%%%%%%%%%%%%%%%% Definiciones básicas %%%%%%%%%%%%%%%%%%%%%%%

\newcommand{\nombreautor}{Ángel Baeza Sánchez}
\newcommand{\nombrecoautor}{Marta Vidal González}
\newcommand{\nombrecoautorespacio}{Marta Vidal González }
\newcommand{\nombreproductor}{Eva María Pérez Lajarín}
\newcommand{\nombretutor}{María Zapata Cáceres }
\newcommand{\titulotrabajo}{Programación de un videojuego para el Desarrollo del Pensamiento Computacional y Divulgación Ecológica}
\newcommand{\escuela}{Escuela Técnica Superior\\de Ingeniería Informática}
\newcommand{\escuelalargo}{Escuela Técnica Superior de Ingeniería Informática}
\newcommand{\universidad}{Universidad Rey Juan Carlos}
\newcommand{\fecha}{Fecha}
\newcommand{\grado}{Grado en Diseño y Desarrollo de Videojuegos}
\newcommand{\curso}{Curso 2023-2024}
\newcommand{\logoUniversidad}{logoURJC.pdf} % logoURJC.eps

%%%%%%%%%%%%%%%%%%%%%%%%%%%%%%%%%%%%%%%%%%%%%%%%%%%%%%%%%%%%%%%%%%%%






%%%%%%%%%%%%%%%%%%%%%%%%% Otras definiciones %%%%%%%%%%%%%%%%%%%%%%%%%%

% Definiciones de colores (para hidelinks)
\definecolor{BlueLink}{rgb}{0.165,0.322,0.745}
\definecolor{PinkLink}{rgb}{0.8,0.22,0.5}
\definecolor{gray}{rgb}{0.6,0.6,0.6}


% Enlaces
\hypersetup{hidelinks,pageanchor=true,colorlinks,citecolor=PinkLink,urlcolor=black,linkcolor=BlueLink}


\newcommand\blankpage{%
    \newpage
    \null
    \thispagestyle{empty}%
    %\addtocounter{page}{-1}%
    \newpage}


% Texto referencias
\addto{\captionsspanish}{\renewcommand{\bibname}{Bibliografía}}

% Texto Índice de tablas
\addto\captionsspanish{
\def\tablename{Tabla}
\def\listtablename{\'{I}ndice de tablas}
}


\floatname{algorithm}{Algoritmo}

\newfloat{algorithm}{t}{lop}

%% Etiquetas de comentarios (tutor/alumno)
\newif\ifdraft
\drafttrue
\usepackage{subcaption}
\input{macros}


%\newenvironment{pseudocodigo}[1][htb]
%  {\renewcommand{\algorithmcfname}{Pseudocódig}% Update algorithm name
%   \begin{algorithm}[#1]%
%  }{\end{algorithm}}
  
%%%%%%%%%%%%%%%%%%%%%%%%%%%%%%%%%%%%%%%%%%%%%%%%%%%%%%%%%%%%%%%%%%%%





%%%%%%%%%%%%%%%%%%%%%%% Estilo de código (en Python) %%%%%%%%%%%%%%%%%%%%%%%

\definecolor{bg}{rgb}{0.95,0.95,0.95}
\definecolor{mydeepteal}{rgb}{0.16,0.22,0.23}
\definecolor{myteal}{rgb}{0.31,0.44,0.46}
\definecolor{mymediumteal}{rgb}{0.41,0.58,0.60}

\DeclareFixedFont{\ttb}{T1}{txtt}{bx}{n}{12} % for bold
\DeclareFixedFont{\ttm}{T1}{txtt}{m}{n}{12}  % for normal


%\newcommand*{\FormatDigit}[1]{\textcolor{mydeepteal}{#1}}
\newcommand*{\FormatDigit}[1]{\textcolor{black}{#1}}

% Python style for highlighting
\newcommand\mypythonstyle{\lstset{
language=Python,
basicstyle=\ttfamily\small,
%basicstyle=\linespread{1.0}\footnotesize\ttm,
otherkeywords={self},             % Add keywords here
keywordstyle=\bfseries\ttfamily\color{myteal},
%keywordstyle=\ttb\color{myteal},
commentstyle=\itshape\color{myteal},
stringstyle=\color{mydeepteal},
emph={MyClass,__init__},          % Custom highlighting
emphstyle=\ttb\color{mydeepteal},    % Custom highlighting style
% Any extra options here
showstringspaces=false,            %
backgroundcolor=\color{bg},
rulecolor = \color{bg},
%identifierstyle=\color{deepgreen},
breaklines=true,
numbers=left,
numbersep=5pt,
numberstyle=\tiny,
tabsize=4,
xleftmargin=1em,
frame = single,
framesep = 3pt,
framextopmargin=0pt,
framexbottommargin=0pt,
framexleftmargin=0pt,
framexrightmargin=0pt,
fontadjust=true,
basewidth=0.55em, % compactness of code
upquote=true,
}}

% Python environment
\lstnewenvironment{mypython}[1][]
{
\mypythonstyle
\lstset{#1}
}
{}

\newcommand\mypythonstylenormalinline{\lstset{
language=Python,
basicstyle=\ttfamily\normalsize,
%basicstyle=\linespread{1.0}\footnotesize\ttm,
otherkeywords={self},            % Add keywords here
keywordstyle=\bfseries\ttfamily\color{myteal},
%keywordstyle=\ttb\color{myteal},
commentstyle=\itshape\color{mymediumteal},
stringstyle=\color{mydeepteal},
emph={MyClass,__init__},          % Custom highlighting
emphstyle=\ttb\color{mydeepteal},    % Custom highlighting style
% Any extra options here
showstringspaces=false,            %
backgroundcolor=\color{bg},
rulecolor = \color{bg},
%identifierstyle=\color{deepgreen},
breaklines=false,
numbers=left,
numbersep=5pt,
numberstyle=\tiny,
tabsize=4,
xleftmargin=0em,
frame = single,
framesep = 3pt,
framextopmargin=0pt,
framexbottommargin=0pt,
framexleftmargin=0pt,
framexrightmargin=0pt,
fontadjust=true,
%basewidth=0.55em, % compactness of code
upquote=true,
}}

\newcommand\mypythoninline[1]{{\mypythonstylenormalinline\lstinline!#1!}}

%%%%%%%%%%%%%%%%%%%%%%%%%%%%%%%%%%%%%%%%%%%%%%%%%%%%%%%%%%%%%%%%%%%%%%%%%%%%%%




%%%%%%%%%%%%%%%%%%%%%%%%%%%% Comandos definidos por el autor 

\newcommand{\transpuesta}{\mbox{\tiny $\mathsf{T}$}}








%%%%%%%%%%%%%%%%%%%%%%%%%%%%%%%%%%%%%%%%%%%%%%%%%%%%%%%%%%%%%%%%%%%%%%%
%                           Inicio del documento                       
%%%%%%%%%%%%%%%%%%%%%%%%%%%%%%%%%%%%%%%%%%%%%%%%%%%%%%%%%%%%%%%%%%%%%%%


\begin{document}

\pagestyle{plain}




%%%%%%%%%%%%%%%%%%%%%%%%%%%%%%%%%%%% Portada %%%%%%%%%%%%%%%%%%%%%%%%%%%%%%%%%%

%\pagenumbering{gobble}
%\pagenumbering{arabic}

% Universidad, Facultad
\begin{titlepage}
\selectlanguage{spanish}


% logo
\begin{center}
    \includegraphics[scale=0.7]{\logoUniversidad}
\end{center}

\bigskip

\begin{center}
\begin{LARGE}
\escuela \\
\end{LARGE}
\end{center}

\bigskip
\bigskip

% Grado
\begin{center}
\begin{large}
\textbf{\grado}\\
\end{large}
\end{center}

% Curso
\begin{center}
\begin{large}
\textbf{\curso}\\
\end{large}
\end{center}

\bigskip

\textbf{\begin{center}
\begin{large}
\textbf{Trabajo Fin de Grado}
\end{large}
\end{center}}

\bigskip
\bigskip
\bigskip

% Nombre del TFG
\begin{center}
\textbf{\begin{large}
\MakeUppercase{\titulotrabajo}\\
\end{large}}
\end{center}

% Nombre del autor
\vspace{\fill}
\begin{center}
\textbf{Autor: \nombreautor}\\ \smallskip
% Tutor
\textbf{Tutor: \nombretutor}\\
% Añadir segundo tutor si hubiera


\bigskip

% Fecha
%\textbf{\fecha}\\
\end{center}
\end{titlepage}


%%%%%%%%%%%%%%%%%%%%%%%% Opcional %%%%%%%%%%%%%%%%%%%%%%
%\blankpage

%\thispagestyle{empty}
%\begin{center}

% Nombre del trabajo
%\textbf{\begin{large}
%\MakeUppercase{\titulotrabajo}\\*
%\end{large}}
%\vspace*{0.2cm}
%\vspace{5cm}

% Nombre del autor y del tutor
%\large Autor: \nombreautor \\* \medskip
%\large Tutor: \nombretutor \\*

%\vfill

% Escuela, universidad y fecha
%\escuelalargo \\ \smallskip
%\universidad \\
%\vspace{1cm}
%\fecha \\

%\clearpage

%\end{center}
%%%%%%%%%%%%%%%%%%%%%%%%%%%%%%%%%%%%%%%%%%%%%%%%%%%%%%%%

\hypersetup{pageanchor=true}

\normalsize
\afterpage{\blankpage} % Se deben añadir página en blanco para que lo capítulos de la memoria o estas secciones introductorias empiecen en páginas impares

%%%%%%%%%%%%%%%%%%%%%%%%%%%%%%%%%%%%%%%%%%%%%%%%%%%%%%%%%%%%%%%%%%%%%%%%%%%%%%%





% Estilo de párrafo de los capítulos
\setlength{\parskip}{0.75em}
\renewcommand{\baselinestretch}{1.25}
% Interlineado simple
\spacing{1}

\pagenumbering{Roman}
\setcounter{page}{2}


%%%%%%%%%%%%%%%%%%%%%%%%% Agradecimientos o dedicatoria %%%%%%%%%%%%%%%%%%%%%%%%%%%

\chapter*{Agradecimientos}

Quisiera agradecer a mis padres y a mi hermana, por su apoyo incondicional durante toda mi vida, también por su excelente sentido del humor, sin el cuál no creo que hubiera llegado (vivo) a donde estoy hoy.

También a mi pareja, Eva, por estar siempre a mi lado pase lo que pase, por su apoyo y cariño infinitos en los mejores y peores días y por soportarme estos últimos meses de estrés para terminar la carrera. Te quiero.

A mis compañeros de piso, Ceci, Javi y Vicente por ser una segunda familia, y por su ayuda constante en mi día a día, sois amigos increíbles.

Y finalmente a mis amigos del grado, sin los cuales no me habría sacado ni el primer examen de primero de carrera, muchas gracias por contestar esas dudas incesantes los veinte minutos antes de cada examen.


\afterpage{\blankpage}

%%%%%%%%%%%%%%%%%%%%%%%%%%%%%%%%%%%%%%%%%%%%%%%%%%%%%%%%%%%%%%%%%%%%%%%%%%%%%%%%%%%






%%%%%%%%%%%%%%%%%%%%%%%%%%%%%%%%%%%% Resumen %%%%%%%%%%%%%%%%%%%%%%%%%%%%%%%%%%%%%%

\chapter*{Resumen}

El Pensamiento Computacional puede definirse como una habilidad cognitiva que permite resolver problemas utilizando estrategias computacionales. 

En los últimos años el Pensamiento Computacional ha ganado popularidad como una habilidad indispensable en la juventud de hoy en día, es tanto así que la mayoría de planes educativos están comenzando a requerir que se enseñe en las aulas. Esta memoria contiene la documentación del proceso de desarrollo de EcoRescue, un videojuego educativo que busca ayudar al desarrollo del Pensamiento Computacional en niños y adolescentes, divulgar acerca del medioambiente y la restauración de ecosistemas, y, sobre todo, ser una experiencia amena y divertida para el aula.

Este documento contiene una descripción en profundidad de: motivaciones, objetivos, análisis de requisitos, validación y conclusiones en lo relativo al proyecto. Además de una seción en la que se analizarán a nivel técnico algunos aspectos de la implementación del videojuego.

\mbox{} \bigskip

\noindent \textbf{Palabras clave}:
\begin{compactitem}
    \item Unity
    \item Pensamiento Computacional
    \item Restauración de Ecosistemas
    \item Desarrollo de Videojuegos
    \item Herramientas de Desarrollo de Videojuegos
    \item C\#
\end{compactitem}


\chapter*{Abstract}


Computational Thinking can be defined as a cognitive ability that allows the resolution of problems by the use of computational strategies.

In the last years, Computational Thinking has gained traction as an indispensable ability in today's youth, much so that the majority of educational plans are starting to require it being teached at schools and highschools. This report contains the documentation of the development process for EcoRescue, an educational videogame which aims to aid with the development of Computational Thinking in students, teach about the environment and ecosystem restoration, as well as being an enjoyable and fun experience.

This document contains an in depth description of the: motivations, objectives, requirement analysis, validation and conclusions pertaining this project. As well as a technical analysis  of some aspects of the videogame implementation.

\mbox{} \bigskip

\noindent \textbf{Keywords}:
\begin{compactitem}
    \item Unity
    \item Computational Thinking
    \item Ecosystem Restoration
    \item Videogame Development
    \item Videogame Development Tools
    \item C\#
\end{compactitem}


\afterpage{\blankpage}


%%%%%%%%%%%%%%%%%%%%%%%%%%%%%%%%%%%%%%%%%%%%%%%%%%%%%%%%%%%%%%%%%%%%%%%%%%%%%%%%%%%





%%%%%%%%%%%%%%%%%%%%%%%%%%%%%%%%%%%% Índices %%%%%%%%%%%%%%%%%%%%%%%%%%%%%%%%%%%%

% Estilo de párrafo de los Índices
\setlength{\parskip}{1pt}
\renewcommand{\baselinestretch}{1}
\renewcommand{\contentsname}{Índice de contenidos}


% Índice de contenidos
\tableofcontents
\afterpage{\blankpage}

% Índice de tablas (OPCIONAL)
\listoftables
\addcontentsline{toc}{chapter}{\noindent \listtablename}

% Índice de figuras (OPCIONAL)
\listoffigures
\addcontentsline{toc}{chapter}{\listfigurename}

% Índice de códigos/algoritmos (OPCIONAL).   El término "Códigos" se puede cambiar por "Métodos", "Funciones", "Algoritmos", etc.
\renewcommand\lstlistlistingname{Códigos}
\renewcommand\lstlistingname{Código}
\renewcommand\lstlistlistingname{Índice de códigos}

\lstlistoflistings
\addcontentsline{toc}{chapter}{\lstlistlistingname}


% En este documento (de momento) no se ha considerado incluir un índice de algoritmos/pseudocódigos, como el que aparece en \ref{AdditionalLouvain}

%%%%%%%%%%%%%%%%%%%%%%%%%%%%%%%%%%%%%%%%%%%%%%%%%%%%%%%%%%%%%%%%%%%%%%%%%%%%%%%%%%%





%%%%%%%%%%%%%%%%%%%%%%% Cabeceras y pies de página (Opcional) %%%%%%%%%%%%%%%%%%%%%%%

%\setlength{\headheight}{15.2pt}
\pagestyle{fancy}


\renewcommand{\chaptermark}[1]{\markboth{Capítulo \thechapter.\ #1}{}}

\pagestyle{fancy}
\fancyhf{}
\fancyhead[LO]{\leftmark}
\fancyhead[RO]{}
\fancyhead[RE]{\nouppercase\rightmark}
\fancyhead[LE]{}
\fancyfoot[C]{\thepage}

%%%%%%%%%%%%%%%%%%%%%%%%%%%%%%%%%%%%%%%%%%%%%%%%%%%%%%%%%%%%%%%%%%%%%%%%%%%%%%%%%%%%






%%%%%%%%%%%%%%%%%%%%%%%%%%%%%% Capítulos de la memoria %%%%%%%%%%%%%%%%%%%%%%%%%%%%%



% Capítulo 1
\chapter{Introducción}
\label{sec:intro}


%%%%%%%%%%%%%%%%%%%%%%%%%%%%%%%%%%%%%%%%%%%%%%%%%%%%%%%%%%%%%%%%%%%%%%%%%%

% Estilo resto de páginas
\pagestyle{fancy}


% Estilo de párrafo de los capítulos
\setlength{\parskip}{0.75em}
\renewcommand{\baselinestretch}{1.25}
% Interlineado simple
\spacing{1}
% Numeración contenido
\pagenumbering{arabic}
\setcounter{page}{1}

%%%%%%%%%%%%%%%%%%%%%%%%%%%%%%%%%%%%%%%%%%%%%%%%%%%%%%%%%%%%%%%%%%%%%%%%%%

\section{Descripción General}
En este documento se desarrollarán las motivaciones, implicaciones y detalles de la implementación de este Trabajo de Fin de Grado.

El objetivo general de este proyecto ha sido el desarrollo de un videojuego educativo en conjunto con otra alumna, \nombrecoautorespacio que se ha encargado de la del diseño y el arte del videojuego.
Este videojuego tiene como meta servir como apoyo para el desarrollo del Pensamiento Computacional (en adelante PC) en niños de Primaria/ESO. El PC se define como una habilidad cognitiva que permite resolver problemas
 utilizando estrategias computacionales\cite{tesismaria}, de las cuales se pueden extraer seis factores y sus definiciones\cite{tesismaria}:
\begin{compactitem}
    \item Abstracción: Proceso de obtener algo simple desde algo complejo, obviando los detalles.
    \item Análisis de datos: Buscar, seleccionar, organizar y analizar lógicamente los datos. 
    \item Descomposición de problemas: Descomponer problemas en otros más pequeños que pueden resolverse con mayor facilidad. 
    \item Algoritmia: Identificar instrucciones específicas y explícitas que, paso a paso, llevan acabo un proceso.
    \item Depuración de errores: Identificar y corregir los errores en la solución aportada.
    \item Generalización: Transferir un proceso de resolución de problemas a una variedad grande de problemas.
\end{compactitem}

Partiendo de esta base, se ha desarrollado el videojuego EcoRescue con el objetivo de añadir elementos jugables que ayuden a potenciar gran parte de todas estas facetas del PC. Dado que este proyecto se ha desarrollado en conjunto con otra alumna, la documentación del diseño queda fuera del alcance de este documento, sin embargo, sí que habrá referencias al Game Design Document (GDD) cuando sea preciso para ilustrar el punto que se esté desarrollando en ese momento.

EcoRescue es un videojuego que además de ser un apoyo en el desarrollo del PC, pretende ser didáctico con los temas que muestra y enseñar a los niños que lo jueguen información acerca del medio ambiente y la restauración de ecosistemas, siendo así la mecánica principal restaurar los ecosistemas en los que consisten sus niveles. Con este fin se ha contado con el apoyo de la Doctoranda en restauración de ecosistemas, \nombreproductor.

Las mecánicas de EcoRescue se pueden definir en dos fases, la de investigación y la de restauración. Al iniciar el juego, el jugador se encuentra con una lista de niveles que muestran una serie de biomas y características a restaurar, una vez entra, comenzará la fase de investigación:
\begin{compactitem}
    \item Durante la fase de investigación, el jugador puede observar un mapa cuadriculado en vista isométrica con varios biomas visibles.
    \item Encima de cada bioma, verá una burbuja con un icono (Figura \ref{fig:burbujas}) con la que deberá interactuar para aprender sobre qué le puede estar ocurriendo al bioma.
    \item Esta interacción muestra un libro con información acerca del bioma, su alteración y sus problemas (Figura \ref{fig:libro}).
    \item El jugador tendrá que completar un minijuego en el que relacione los problemas con sus posibles consecuencias (Figura \ref{fig:relations}) para demostrar que ha entendido qué es lo que le está ocurriendo al bioma.
\end{compactitem}

\begin{figure}[H]
  \centering
	\includegraphics[width=350px,clip=true]{burbujas.png}
  \caption{Mapa con burbujas de bioma visibles}
  \label{fig:burbujas}
\end{figure}

\begin{figure}[H]
  \centering
	\includegraphics[width=350px,clip=true]{libro_info.png}
  \caption{Libro con información de problemas medioambientales}
  \label{fig:libro}
\end{figure}

\begin{figure}[H]
  \centering
	\includegraphics[width=350px,clip=true]{libro_relaciones.png}
  \caption{Libro de Relaciones Problema - Consecuencia}
  \label{fig:relations}
\end{figure}

Una vez el jugador haya demostrado que ha entendido todas las alteraciones y problemas de todos los biomas, el juego pasará a la segunda fase, la de restauración.
\begin{compactitem}
    \item Durante la fase de restauración, el jugador tendrá un presupuesto de energía para gastar en máquinas restauradoras.
    \item Cada bioma tendrá una alteración, que a su vez estará compuesta de 2-3 fases, cada una definida por un problema distinto (Deforestación, Desertificación, Sobrepesca…)
    \item El jugador deberá comprar máquinas restauradoras de la tienda y colocarlas en el bioma para que puedan hacer efecto, los distintos efectos de estas restaurarán el ecosistema de una forma u otra.
    \item La jugabilidad radica en que el jugador tendrá distintas opciones de máquinas (tres) por cada fase de cada bioma, y cada opción tendrá un efecto distinto, teniendo en cuenta la información que ha adquirido durante la primera fase, y la información que proporciona la tienda acerca del efecto de la máquina, el jugador deberá hacer una elección que le permita restaurar el ecosistema por un coste apropiado, con el objetivo de no quedarse sin presupuesto antes de terminar todas las fases de todos los biomas.
\end{compactitem}

Teniendo en cuenta las estrategias del PC y las mecánicas del juego, podemos asimismo definir una tabla (Tabla \ref{fig:tablaPCMecanicas}) que las relacione entre sí, dejando claro de qué forma se pretende ayudar a potenciar el PC en los jugadores.
\raggedbottom
\begin{table}[H]
\begin{center}
\setlength{\tabcolsep}{5pt}
\renewcommand{\arraystretch}{1.2}
\begin{tabular}{ | m{8em} | m{30em} | } 
  \hline
  Estrategia & Mecánica \\ 
  \hline
  Abstracción & A la hora de realizar los diagramas de correlación entre indicadores (datos) y sus consecuencias, se reduce al mínimo imprescindible la explicación del contexto de la alteración del ecosistema. \\ 
  \hline
  Análisis de Datos & La fase de investigación muestra una serie de datos que posteriormente el jugador analiza para poder correlacionarlos con consecuencias. \\ 
  \hline
  Descomposición de problemas & Un nivel está visiblemente desglosado en diferentes tareas y subtareas. Para completar el nivel es necesario restaurar todos los biomas y para restaurar cada bioma hay que finalizar todas las fases del plan de restauración (que pueden considerarse tareas a cumplimentar). \\ 
  \hline
  Algoritmia & A la hora de restaurar un ecosistema, está todo claramente separado de forma secuencial. Incluso el propio nivel se desarrolla de forma secuencial. \\ 
  \hline
  Depuración de Errores & Es posible cometer errores y es importante detectarlos y subsanarlos a tiempo. La idea es dejar cierto margen para que los jugadores puedan echarse atrás y arreglar los fallos. \\ 
  \hline
    & El jugador tiene recursos (moneda) limitados y debe hacer un buen uso de ellos para poder completar los niveles, si se queda sin dinero, deberá volver a intentarlo para poder minimizar el gasto. \\ 
  \hline
  Generalización & A menudo las alteraciones de los ecosistemas tienen factores en común, a lo largo del progreso en el videojuego los jugadores podrán ir discerniendo que existen muchas correlaciones entre alteraciones y sus consecuencias. Por ejemplo: Si hay un problema con el suelo (del tipo que sea) la flora no va a poder prosperar correctamente y si la flora no está en sus condiciones óptimas, la fauna tampoco lo estará. \\ 
  \hline
\end{tabular}
\centering
\caption{Relaciones entre Estrategias del PC con mecánicas del videojuego.}
\label{fig:tablaPCMecanicas}
\end{center}
\end{table}

\section{Motivación}

La motivación para el desarrollo de EcoRescue nace como respuesta al Plan de Acción de Educación Digital 2021-2027 de la Comisión Europea\cite{europaPlan}, además de verse reforzada en 
España por la modificación de la Ley Orgánica 3/2020 del 3 de diciembre de 2020\cite{lomce}, y los Reales Decretos 95/2022\cite{boeInfantil}, 157/2022\cite{boePrimaria}, 217/2022\cite{boeSecundaria} y 243/2022\cite{boebachillerato}, del 1 de febrero, 1 y 19 de marzo y 5 de abril del 2022 respectivamente, 
que establecen las enseñanzas mínimas de la educación, incluyendo el PC, en las etapas de Educación Infantil, Primaria y Secundaria Obligatoria y Bachillerato, como una competencia transversal en diversas asignaturas o en asignaturas determinadas.

Como se ha mencionado anteriormente, además de servir para desarrollar el PC, uno de los objetivos del proyecto es concienciar sobre la importancia de cuidar el medioambiente y de como distintos indicadores y alteraciones pueden afectar negativamente a estos. De esta forma, el juego puede ayudar a niños y adolescentes a relacionar consecuencias del sistema productivo de la sociedad actual con los daños que sufren actualmente nuestros ecosistemas y entornos naturales. A su vez, ilustra de forma eficiente como interaccionan diferentes ecosistemas entre sí y como se relacionan sus diferentes elementos: fauna, flora y el soporte natural. Además, el juego aporta una versión optimista de como entornos que se consideran destruidos se pueden recuperar mediante la intervención y tecnología adecuados. 

Este enfoque es relevante dado que según los Reales Decretos mencionados anteriormente\cite{boeInfantil}\cite{boePrimaria}\cite{boeSecundaria}\cite{boebachillerato} se estipula que en lo relacionado al Conocimiento del Medio, Biología y Geología se buscará fomentar el razonamiento y pensamiento crítico (computacional) para resolver problemas o dar explicación a procesos de la vida cotidiana.

Finalmente habría que considerar también la motivación adicional de aprender el uso de nuevas tecnologías mediante el desarrollo de un proyecto, que tendría como objetivo el propio aprendizaje y la acumulación de experiencia de llevar un proyecto desde su etapa de concepción hasta validarlo con un grupo de alumnos de instituto.

\raggedbottom
\section{Objetivos}
\subsection{Objetivos generales}
El objetivo general del trabajo es la elaboración de un prototipo funcional del videojuego EcoRescue para el desarrollo del PC y la concienciación en la restauración de ecosistemas, 
en el que los jugadores puedan acceder a niveles generados procedimentalmente, compuestos por un número
de 1-N biomas. Posteriormente deberán poder informarse de estos, relacionar problemas y consecuencias y finalmente 
colocar máquinas en los biomas para restaurar los distintos problemas de cada alteracion.

El prototipo deberá tener interfaces funcionales que muestren la información de los niveles, biomas, alteraciones, problemas, consecuencias, máquinas y tutoriales (Figura \ref{fig:UI}). Los tutoriales vendrán dados por diálogos que saldrán del 
icono de un robot (Figura \ref{fig:robot}) que guiará al jugador durante su experiencia de juego.

\begin{figure}[H]
    \centering
      \includegraphics[width=350px,clip=true]{interfaz_restauracion.png}
    \caption{Interfaz general de EcoRescue}
    \label{fig:UI}
\end{figure}

\begin{figure}[H]
    \centering
      \includegraphics[width=350px,clip=true]{convo_robot.png}
    \caption{Conversación de tutorial del robot}
    \label{fig:robot}
\end{figure}

El prototipo también deberá ser jugablemente disfrutable, esto implica desarrollar sistemas de control, sonido y animaciones que hagan que el juego sea vistoso y satisfactorio de jugar.

\subsection{Generación procedimental}

Una parte importante de EcoRescue es la facilidad de creación de contenido, para que desarrollar niveles sea más sencillo, el juego se ha enfocado desde un principio con la generación procedimental en mente, de forma que los niveles deberán generarse en base a algoritmia\cite{FastNoiseLite}, y cosas como las alteraciones en un nivel o el percentil al que se debe completar una fase deberán del mismo modo seguir un patrón procedimental.

\subsection{Herramientas del desarrollo}

De cara a la generación procedimental de niveles, el prototipo necesitará herramientas para que enlazar el gran volumen de contenido requerido por este enfoque de desarrollo con las estructuras internas del videojuego sea sencillo y mayormente automatizado. 

De esta forma, se requiere desarrollar un script que importe automáticamente todo el contenido (biomas, alteraciones, problemas, consecuencias, sprites, modelos...) desde un fichero de excel a un formato legible por el juego.

Además, también se requiere la creación de herramientas que permitan elegir de qué forma se quiere que se presente un nivel, incluyendo: personalización de las reglas de la generación procedimental de mapas, selección de Biomas por nivel y selección de Alteraciones por Bioma.

Además, el prototipo necesitará un sistema de generación de diálogos para que el robot de los tutoriales pueda comunicarse efectivamente con el jugador.

\subsection{Recogida de datos}

De cara a la medición de resultados, es necesario que el prototipo sea capaz de comunicarse con una base de datos para poder monitorizar las partidas de los jugadores y así poder interpretar sus movimientos. 
Esta comunicación con la base de datos debería recoger los datos básicos del usuario (nombre, edad...) (Figura \ref{fig:datos}) así como datos sobre su partida (máquinas colocadas, relaciones correctas, veces que ha perdido el nivel...).

\begin{figure}[H]
    \centering
      \includegraphics[width=350px,clip=true]{datos_pj.png}
    \caption{Pantalla de introducción de datos personales}
    \label{fig:datos}
\end{figure}

\subsection{Validación}

El objetivo más importante a nivel de proyecto es validar el prototipo frente a un grupo de alumnos de 1º de la ESO reales, con el objetivo de poder utilizar los datos de una encuesta así como los datos obtenidos en la base de datos para poder extraer conclusiones acerca de la efectividad del prototipo como herramienta didáctica para el desarrollo del PC, además de como herramienta divulgativa en lo referente a la restauración de ecosistemas.
Queda patente, pues, que los objetivos de la validación deben ser:
\begin{compactitem}
    \item Que el juego verdaderamente sirva para desarrolar el PC.
    \item Que el juego divulge información interesante acerca de la restauración de ecosistemas.
    \item Que el juego resulte una expriencia amena y divertida, a fin de que los alumnos disfruten del aprendizaje.
\end{compactitem} 

\subsection{Resumen de Objetivos}

En resumen, el el proyecto pretende, como objetivo principal desarrollar un juego completo y funcional que permita al jugador informarse sobre la restauración de ecosistemas y actuar en consecuencia.

Además de los objetivos secundarios:
\begin{compactitem}
  \item Desarrollar el PC del jugador en el aula.
  \item Desarrollar un juego que sea divertido y ameno.
  \item Desarrollar un juego con niveles generados procedimentalmente, además de herramientas que permitan crear contenido de forma eficiente para este.
  \item Recabar información sobre las partidas de los jugadores con el objetivo de interpretar sus movimientos y extraer conclusiones sobre la efectividad del prototipo.
\end{compactitem}

\section{Metodología}

Para el desarrollo del prototipo se ha utilizado una metodología de trabajo iterativa-incremental enfocada en la revisión de objetivos y
avances entre el tutor y los alumnos. Durante una serie de revisiones se ha evaluado el estado del proyecto y se han marcado objetivos de cara a las siguientes revisiones. El desarrollo se puede dividir en tres fases: 
\begin{itemize}
	\item Fase 1: Pre-Producción y Análisis de Requisitos, Junio 2023 - Diciembre 2023
	\item Fase 2: Producción, Enero 2024 - Mayo 2024
	\item Fase 3: Polish y Validación, Junio 2024 
\end{itemize}

La primera fase se utilizó como una ventana de tiempo en la que prototipar los controles y la generación procedimental del mapa, además de realizar un análisis de requisitos y un documento de diseño exhaustivo para tener claro qué clase de prototipo realizar. Este documento de diseño se fue iterando con \nombretutor de cara a la producción final.

Durante la segunda fase se realizó el groso de la producción del juego, con el diseño ya fijado, esta fase consistió sobre todo en implementar el bucle de juego general y definir el contenido que iba a estar presente en el juego final.

La tercera y última fase consistió en la creación de herramientas para el desarrollo de contenido/niveles, pulir el juego añadiendo pequeñas animaciones, efectos y sonidos, implementar un sistema de recogida de datos en tiempo real via una base de datos remota y finalmente acudir a un instituto para poder validar el prototipo con alumnos de verdad.

Esta metodología ha acabado siendo todo un exito con este prototipo, ya que ha permitido un avance progresivo donde no ha acabado habiendo esfuerzo malgastado, esto ha permitido que el desarrollo del prototipo haya sido sólido y no se hayan requerido cambios radicales.

% \afterpage{\blankpage} % puede generar problema en índice de contenidos
% \newpage

% Capítulo 2
\chapter{Estado del Arte}
\label{sec:estadodelarte}

\section{Chirimoya}


aa aaa aaa aaa aaa aaa aaa aaa aaa aaa aaa aaa aaa aaa a
aa aaa aaa aaa aaa aaa aaa aaa aaa aaa aaa aaa aaa aaa aaa aaa a
aa aaa aaa aaa aaa aaa aaa aaa aaa aaa aaa aaa aaa aaa aaa a

aa aaa aaa aaa aaa aaa aaa aaa aaa aaa aaa aaa aaa aaa aaa aaa a
aa aaa aaa aaa aaa aaa aaa aaa aaa aaa aaa aaa aaa aaa aaa aaa aaa a
aa aaa aaa aaa aaa aaa aaa aaa aaa aaa aaa aaa aaa aaa aaa aaa a

aa aaa aaa aaa aaa aaa aaa aaa aaa aaa aaa aaa aaa aaa aaa aaa aaa aaa a
aa aaa aaa aaa aaa aaa aaa aaa aaa aaa aaa aaa aaa aaa aaa aaa aaa a

aa aaa aaa aaa aaa aaa aaa aaa aaa aaa aaa aaa aaa aaa aaa aaa aaa aaa a

aa aaa aaa aaa aaa aaa aaa aaa aaa aaa aaa aaa aaa aaa a
aa aaa aaa aaa aaa aaa aaa aaa aaa aaa aaa aaa aaa aaa aaa aaa a
aa aaa aaa aaa aaa aaa aaa aaa aaa aaa aaa aaa aaa aaa aaa a

aa aaa aaa aaa aaa aaa aaa aaa aaa aaa aaa aaa aaa aaa aaa aaa a
aa aaa aaa aaa aaa aaa aaa aaa aaa aaa aaa aaa aaa aaa aaa aaa aaa a
aa aaa aaa aaa aaa aaa aaa aaa aaa aaa aaa aaa aaa aaa aaa aaa a

aa aaa aaa aaa aaa aaa aaa aaa aaa aaa aaa aaa aaa aaa aaa aaa aaa aaa a
aa aaa aaa aaa aaa aaa aaa aaa aaa aaa aaa aaa aaa aaa aaa aaa aaa a

% Capítulo 3
\chapter{Descripción Informática}
\label{sec:descripcionInformatica}

\section{Requisitos}

En esta sección se abordarán los requisitos planteados en la aplicación, tanto los iniciales como los que se han ido planteando a lo largo del desarrollo ya sean funcionales o no funcionales. 

\subsection{Requisitos Funcionales}
\begin{itemize}
    \item RF1. Como usuario puedo rellenar mi nombre, edad y género.
    \item RF2. Como usuario puedo interactuar con las burbujas de cada bioma.
    \item RF3. Como usuario puedo informarme acerca de los alteraciones y problemas de los biomas desde el libro informativo.
    \item RF4. Como usuario puedo relacionar problemas con consequencias y ser informado de si una relación es correcta.
    \item RF5. Como usuario puedo pasar de la fase de investigación a la fase de restauración.
    \item RF6. Como usuario puedo ver en la tienda distintas máquinas restauradoras con distinos efectos.
    \item RF7. Como usuario puedo colocar una máquina y ver que tiene efecto sobre el bioma.
    \item RF8. Como usuario puedo vender esa máquina si opino que no ha funcionado como esperaba.
    \item RF9. Como usuario puedo reiniciar una fase si una máquina destruye el ecosistema.
    \item RF10. Como usuario puedo reiniciar un nivel si me quedo sin dinero u opino que he cometido un error crítico.
    \item RF11. Como usuario puedo abrir el libro informativo en cualquier momento para consultar la información y relaciones de los problemas.
    \item RF12. Como usuario puedo ver el progreso actual de cada fase de cada bioma.
    \item RF13. Como usuario puedo terminar una fase colocando suficientes máquinas restauradoras correctamente.
    \item RF14. Como usuario puedo terminar un nivel una vez he terminado todas las fases de todos los biomas.
\end{itemize}

\subsection{Requisitos No Funcionales}

\begin{itemize}
    \item RNF1. El videojuego tiene que ser divertido.
    \item RNF2. El videojuego tiene que tener una usabilidad mínima.
    \item RNF3. El videojuego debe ser divulgativo en lo referente al a restauración de ecosistemas.
    \item RNF4. El videojuego debe desarrollar el PC
    \item RNF5. El videojuego debe tener herramientas que ayuden al desarrollo de contenido
    \item RNF6. El videojuego debe ser multiplataforma (PC, Tablet, Móvil, Web).
\end{itemize}

\subsection{Requisitos de Diseño}

\begin{itemize}
    \item RD1. Cada nivel del videojuego debe tener 1-N biomas posibles.
    \item RD2. Cada bioma debe tener dos alteraciones posibles.
    \item RD3. Al iniciar el nivel, se deberá elegir una alteración de las dos posibles por cada bioma.
    \item RD4. Cada alteración deberá tener 1-3 problemas relacionadas (Figura \ref{fig:diagramaalt}).
    \item RD5. Cada problema deberá tener de 0-2 problemas y de 1-4 consecuencias relacionados.
    \item RD6. Cada problema deberá tener 2-3 máquinas relacionadas.
\end{itemize}

\begin{figure}[H]
    \centering
      \includegraphics[width=350px,clip=true]{diagramaalteracionproblemas.png}
    \caption{Diagrama que muestra la relación entre problemas y alteraciones}
    \label{fig:diagramaalt}
\end{figure}

\section{Herramientas y Tecnologías}

\subsection{C\# \& Unity}

Se ha utilizado Unity\cite{unity} como motor para el desarrollo de EcoRescue y C\#\cite{csharp} como lenguaje de programación, se ha elegido este motor dado que era el motor público con el que los desarrolladores teníamos más experiencia.

\subsection{Visual Studio}

Visual Studio\cite{visualstudio} es un IDE y editor de código para C++ y .Net y C\#, es el editor de código por defecto de Unity y el que se ha utilizado por defecto para el desarrollo del proyecto. 

\subsection{Git \& Github}

Github\cite{github} es un entorno de desarrollo colaborativo y control de versiones web basado en la tecnología Git. Para mantener el proyecto y poder trabajar en el desde distintos equipos, se ha alojado el proyecto de EcoRescue\cite{Repo} en él.

\subsection{FastNoiseLite}

La generación procedimental de los mapas de los niveles del prototipo utilizan texturas de ruido perlin generadas automáticamente por la librería FastNoiseLite\cite{FastNoiseLite}. Esta es una librería ligera, optimizada y muy sencilla de usar que permite obtener patrones aleatorios altamente personalizables de una forma muy cómoda y sencilla.

\subsection{Adobe XD \& AkyuiUnity}

Adobe XD\cite{xd} es un programa que forma parte de la 'Suite' creativa de Adobe, permite a los usuarios el desarrollo de interfaces basado en iconos y formas vectoriales. \nombrecoautorespacio ha utilizado este programa para desarrollar las interfaces e iconos del videojuego. Esta herramientas se ha aprovechado enormemente mediante el uso de AkyuiUnity\cite{AkyuiUnity} y AnKuchen\cite{AnKuchen}, dado que el conjunto de estas dos librerías permiten establecer un flujo de trabajo mediante el cual se pueden importar los proyectos de Adobe XD como prefabs dentro de Unity, además de auto generar clases para poder acceder a todos los elementos del prefab a partir de su ID. Este 'workflow' ha permitido que el desarrollo e implementación de interfaces sea muy dinámico y fluido. 

\subsection{UniTask}

UniTask\cite{UniTask} es una librería que ofrece una implementación de async/await sin necesidad de alocataciones de memoria basada en estructuras. Se ha aprovechado esta librería para la implementación de varias rutinas de comportamiento del prototipo, desde eventos de UI hasta los propios controles del juego.

\subsection{DUJAL}

DUJAL\cite{DUJAL} es una librería de creación propia para otro Trabajo de Fin de Grado, esta librería contiene utilidades como un sistema de diálogos basado en grafos, un sistema de sonido estático y accesible desde todo el proyecto, llamadas estáticas a sacudidas de cámara, controladores de personajes y más. En este proyecto se ha importado solamente los módulos de diálogo, sonido y guardado de partida.

\subsection{Audacity}

Audacity\cite{Audacity} es un programa de edición y grabación de audio, todos los sonidos del videojuego se han obtenido retocando samples de uso libre utilizando Audacity.  

\subsection{Notion}

Notion\cite{notion} es una plataforma para la creación de documentos, wikis, listas de tareas y más. Para el desarrollo del proyecto se ha utilizado Notion como base de la documentación para almacenar toda la información referente al diseño, implementación, contenido y referencias artisticas y mecánicas.

\subsection{MariaDB \& SQL}

Para el almacenamiento de datos acerca de los jugadores y las partidas, se ha utilizado una base de datos proporcionada por la URJC basada en MariaDB\cite{mariadb}, un modelo de base de datos relacional que sigue el estándar SQL.


\section{Arquitectura y Análisis}
\subsection{Clases 'Handler' y diseño con enumerados}
EcoRescue está implementado de forma que todo el juego se pueda gestionar a partir de clases de tipo 'Singleton' que contienen referencias a tipos de datos que contienen el estado actualizado de la partida. Estos managers estáticos se encargan de gestionar una parte concreta del juego y sus funciones relacionadas.

Para entender cómo funcionan estas clases gestoras, primero hay que explicar el patrón de diseño que se ha utilizado de cara a organizar el gran volumen de datos necesario para hacer un juego de este tipo de forma procedimental. Este patrón consiste en utilizar enumeraciones para identificar inequívocamente a un 'ScriptableObject' que representa un elemento de juego, para el ejemplo podemos utilizar una 'Machine'. Una 'Machine' tendrá asociado un 'MachineType' que deberá ser único para este 'ScriptableObject' en particular. De esta forma, cuando se esté tratando esta máquina y no se necesite acceder a nada que no sea su nombre, se puede utilizar este enumerado para identificarla. 

Este patrón satisface dos problemas a resolver, el primero es poder utilizar el nombre del enumerado como string en un archivo de tipo excel de cara a poder autogenerar los enumerados de C\#, de forma que el contenido se importe automáticamente mediante un MenuItem de Unity, el segundo motivo es que permite ahorrar memoria a la hora de crear estructuras que necesiten contener listas de ScriptableObject, dado que se estaría guardando únicamente el valor del enumerado, en lugar de tener que hacer copias del propio archivo del ScriptableObject para poder utilizar los valores. Para acceder a los valores que contiene el 'ScriptableObject' se almacena en la clase gestora un diccionario que relaciona el valor del enumerado con el 'ScriptableObject' almacenado en memoria. 

\subsection{Biome, Machine, Relation y Phase 'Handlers'}

Estas clases (Figura \ref{fig:handlerUML}) utilizan todas el patrón mencionado anteriormente, 'BiomeHandler' y 'MachineHandler' tienen diccionarios que contienen relaciones entre Biomas y Máquinas con sus respectivos enumerados, además de una lista de todos los enumerados posibles para el nivel actual. 'RelationHandler' es ligeramente más compleja en el sentido de que tiene tres sets de listas y diccionarios, para alteraciones, problemas y consecuencias. Todas estas clases se poblan en base a los datos que se reciben del nivel seleccionado (contenido en el 'ScriptableObject' de la clase 'Level'). De forma que en todo momomento contienen la información actualizada del nivel actual, que puede ser consultada por la lógica del juego cuando es requerida. 

Cabe destacar que cada una tiene funciones especializadas, ya sea el 'MachineHandler' para guardar información de un diccionario que relacione las máquinas colocadas en el tablero de juego con la casilla sobre la que se encuentran colocadas en ese momento. O BiomeHandler, que relaciona cada bioma con todas las casillas asociadas a este.
  
\begin{figure}[H]
    \centering
      \includegraphics[width=350px,clip=true]{Handler_Class_Diagram.png}
    \caption{Diagrama de clases de los Handlers y ResourceGame}
    \label{fig:handlerUML}
\end{figure}


Por último, el 'BiomePhaseHandler' es el que se encarga de gestionar toda la progresión de la fase de restauración, contiene diccionarios que relacionan todos los problemas que hay en el nivel actual por cada bioma y todas las máquinas posibles por cada problema. Además de diccionarios que relacionan el percentil (Figura \ref{fig:progresion}) mínimo de compleción por fase, la compleción actual por fase o la fase actual por bioma. Esta clase se encarga de registrar si al colocar una máquina esta restaura correctamente el problema relacionado o si en su lugar destruye el ecosistema y fuerza el reinicio de esa fase en ese bioma, y también se encarga de comprobar cuándo se pasa de fase así como de comprobar si todas las fases se han completado.

\begin{figure}[H]
    \centering
      \includegraphics[width=350px,clip=true]{progresion.png}
    \caption{Interfaz de progresion de la fase de restauración}
    \label{fig:progresion}
  \end{figure}

\subsection{ResourceGame y Level}

'ResourceGame' es la clase que gestiona las fases y los datos de un nivel, este lee todos los niveles que existan dentro de la carpeta de 'Levels' y carga el que el botón del selector de niveles le pase por argumento.

La clase 'Level', por otro lado, contiene las funciones que poblan todos los 'Handlers', y que gestionan el flujo de un nivel.
\begin{itemize}
    \item InitPreLevel - 'Cachea' todos los datos del Level en las estructuras de datos contenidas en los 'Handlers'.
    \item InitBubblePhase - Instancia las burbujas que abren el libro de relaciones encima de cada bioma.
    \item InitRelationLevel - Inicializa el libro de relaciones y espera a que se hayan completado todas las relaciones de todos los biomas.
    \item InitGameplayLevel - Habilita la tienda y pasa al modo colocar máquinas, se queda así hasta que se restauran todos los ecosistemas.
\end{itemize}

La clase Level también dobla como herramienta de desarrollo, ya que al ser un 'ScriptableObject' permite al diseñador modificar sus propiedades desde el editor. En este caso, se ha añadido una serie de reglas y opciones de personalización que permiten al diseñador elegir qué clase de reglas quiere que la generación procedimental siga. También permite seleccionar qué biomas van a estar presentes en ese nivel. 

\subsection{Ground \& Generación Procedimental}

La clase 'Ground' se encarga de instanciar todas las casillas del mapa siguiendo una serie de reglas definidas en la clase Level. La generación procedimental utiliza una matriz discretizada a partir de un mapa de ruido perlin generado mediante la librería de FastNoiseLite\cite{FastNoiseLite}. La altura del mapa perlin se utiliza para definir qué bioma va en cada lugar, para asegurar que el juego no es injusto se aplican una serie de cupos de casillas mínimas por cada bioma, si el mapa generado no lo cumple, se regenera.

Cada casilla contenida en Ground tiene asignado un componente de tipo GroundTile, con información relevante acerca del bioma y coordenadas discretas de esa casilla en concreto. 

\subsection{PlayerCurrencyManager}

La clase 'PlayerCurrencyManager' es una clase sencilla que solamente se encarga de mantener actualizado el presupuesto de energía del jugador en todo momento. Gestiona las transacciones de compra y venta de máquinas durante la fase de restauración.

\subsection{Interfaces, XD y AnKuchen}

Todas las interfaces del juego se han ilustrado en Adobe XD y convertido en prefabs utilizando AkyuiUnity\cite{AkyuiUnity}, que permite generar prefabs a partir de la definición de formato de interfaz Ankyui desarrollada por kyubuns. Akyui.Xd a su vez permite convertir los archivos en formato de AdobeXD en archivos de definición de formato de Akyui, de forma que se genera un workflow Adobe XD - Unity directo.

Mediante el uso de AnKuchen\cite{AnKuchen}, también desarrollado por kyubuns, se pueden autogenerar unas clases 'template' a partir de las estructuras en el arbol de la escena de Unity mediante el uso de un componente UICache (Figura \ref{UIuml}), estas 'templates' permiten al programador acceder a los elementos de interfaz ('scrollbar', 'button', 'slider') desde código sin necesitar de preocuparse por referencias o jerarquías. Esta librería, en conjunto con UniTask\cite{UniTask} permite definir comportamientod de interfaces de forma muy cómoda.

\begin{figure}[H]
    \centering
      \includegraphics[width=350px,clip=true]{UI_Class_Diagram.png}
    \caption{Diagrama de clases de la UI del juego}
    \label{UIuml}
\end{figure}

\subsection{Cargas, Audio y Diálogos}

EcoRescue utiliza un sistema de carga asíncrona que lanza una pantalla de carga mientras se cargan cosas por detrás, además de un sistema de gestión de audio y una herramienta de creación de diálogos basada en grafos (Figura \ref{fig:dialogue}). Sin embargo estas herramientas son parte de DUJAL\cite{DUJAL} y quedan fuera del alcance de este documento. 

\begin{figure}[H]
    \centering
      \includegraphics[width=350px,clip=true]{dialogue.png}
    \caption{Herramienta de generación de diálogos}
    \label{fig:dialogue}
\end{figure}

\section{Diseño e Implementación} 

\subsection{Controles}

El componente de control de EcoRescue es una de las partes de la implementación que más tiempo ha llevado, esto es por que el objetivo era que fuese lo más satisfactorio e intuitivo posible. De esta forma, los componentes que gestionan el poder colocar y mover máquinas (Figura \ref{fig:controlUML}) están organizados en jerarquías de herencia que permiten controlar la forma en la que se interactúa con estas de forma modular y sencilla. Las PlaceableMachines, que son las instancias de las máquinas que se pueden colocar, heredan del componente Draggable (Que a su vez hereda de Hoverable). 
\begin{itemize}
    \item Hoverable gestiona el hecho de poder pasar el ratón/dedo por encima de una entidad y lanzar un callback cuando ese evento ocurra. 
    \item Draggable mantiene ese comportamiento, pero también tiene funcionalidad para poder arrastrar el propio GameObject
    \item PlaceableMachine permite invocar la interfaz contextual que se encarga de permitir mover, colocar y vender las máquinas, además de tener una copia profunda del ScriptableObject de la máquina a partir de la cuál se ha creado. 
\end{itemize}

\begin{figure}[H]
    \centering
      \includegraphics[width=350px,clip=true]{Controls_Class_Diagram.png}
    \caption{Diagrama de clases de las clases de control del juego}
    \label{fig:controlUML}
\end{figure}

Además, para poder añadir comportamiento de resaltado a las casillas y máquinas, se ha añadido un componente de tipo Highlighteable a todos los componentes de tipo Hoverable, de forma que se puede modificar el shader y el material del MeshRenderer utilizando el string asociado al tipo de 'Highlight' que corresponda.

Los componentes de tipo Hoverable, además, tienen una referencia (obligatoria) a un componente Selectable, cuya implementación permite lanzar un callback cuando se interactúe con el ratón o el dedo con dicho Hoverable. Esto permite al componente PlaceableMachine suscribirse al evento de 'Click' del Selectable para hacer funcionar la lógica de las máquinas.

Es destacable también que todas las máquinas tienen asociado un componente SnapToGrid (Código \ref{alg:snaptogrid}), que obliga a la posición del PlaceableMachine a discretizarse a los parámetros dados, en este caso unos idénticos a los que sigue la cuadrícula del mapa.

\begin{mypython}[caption={Update loop de la clase SnapToGrid.},label={alg:snaptogrid}]
private void Update()
{
    Vector3 position;

    if (_localPosition)
        position = transform.localPosition;
    else
        position = transform.position;

    if (XcellSize != 0)
        position.x = Mathf.RoundToInt(position.x / XcellSize) * XcellSize;

    if (_snapToHeight && YcellSize != 0 && !_doFixHeight)
        position.y = Mathf.RoundToInt(position.y / YcellSize) * YcellSize;
    else
        position.y = FixedHeight;

    if (ZcellSize != 0)
        position.z = Mathf.RoundToInt(position.z / ZcellSize) * ZcellSize;

    Vector3 sourcePosition;
    Vector3 targetPosition;

    if (_localPosition)
        sourcePosition = transform.localPosition;
    else
        sourcePosition = transform.position;

    if (_doFixHeight) sourcePosition.y = FixedHeight;

    if (!_smooth) _lerpSpeed = 0;
    else _lerpSpeed = _smoothSpeed;
    if (Draggable.IsDragging) _lerpSpeed = 10000000;
    targetPosition = Vector3.SmoothDamp(sourcePosition, position, ref _vel, Time.deltaTime * _lerpSpeed);
    if (_localPosition)
        transform.localPosition = targetPosition;
    else
        transform.position = targetPosition;
}
\end{mypython}

\subsection{Máquinas, Restricciones y Tienda}

A la hora de colocar las máquinas en el tablero de juego hay que tener una serie de elementos a tener en cuenta. Las máquinas tienen dos atributos que pueden variar dependiendo del tipo de máquina: la restricción y el rango. Hay dos tipos de restricciones y dos tipos de rango.

Restriciones:
\begin{compactitem}
    \item Restricción de conteo: Solo se pueden poner máximo N máquinas de este tipo, poner una más del máximo provocará que el ecosistema sea destruído y se tenga que reiniciar la fase.
    \item Restricción de probabilidad: Usar este tipo de máquina suele resultar en un gran aumento percentil de la restauración de un problema concreto, sin embargo, hay una pequeña, media o gran probabilidad de que usar una de estas máquinas destruya automáticamente el bioma.
\end{compactitem}

Las restricciones se gestionan todas en base a un enumerado del tipo de restriccion y reciben un entero que dictamina o bien el porcentaje de probabilidad de que destruyan el ecosistema, o bien la cantidad de veces que puede colocarse una máquina.

Rangos:
\begin{compactitem}
    \item Rango infinito (Figura \ref{fig:rango_inf}) - La máquina afecta a absolutamente todo el bioma.
    \item Rango por forma (Figura \ref{fig:rango_form}) - La máquina afecta proporcionalmente al bioma en base número de casillas que afecta por su rango sobre el número de casillas totales del bioma.
\end{compactitem}

\begin{figure}[H]
    \centering
      \includegraphics[width=350px,clip=true]{rango_infinito.png}
    \caption{Máquina de rango infinito}
    \label{fig:rango_inf}
  \end{figure}

  \begin{figure}[H]
    \centering
      \includegraphics[width=350px,clip=true]{rango_chiquito.png}
    \caption{Máquina de rango por forma}
    \label{fig:rango_form}
  \end{figure}

La implementación de las máquinas con rango por forma es destacable dado que se utiliza una textura en blanco y negro (Figura \ref{fig:rango_pattern}) con la forma del rango que se quiera utilizar y esta se cachea al inicializar la máquina (Código \ref{alg:matrizmachine}) (durante la carga del nivel) en forma de una matriz bidimensional de ceros y unos, donde los unos son las casillas sobre las que actúa la máquina.
Esta implementación es bastante eficiente y permite limpiar las casillas que no son afectadas por la máquina a la vez que resaltan las que si lo son sin necesidad de hacer pasadas extra.

\begin{figure}[H]
    \centering
      \includegraphics[width=350px,clip=true]{Cross_MachinePattern.png}
    \caption{Ejemplo de textura usado para el rango de una máquina}
    \label{fig:rango_pattern}
  \end{figure}

\begin{mypython}[caption={Código para cachear la matríz del rango de una máquina.},label={alg:matrizmachine}]
private void PrecalcTexture() 
{
    if (_patternTexture == null) return;
    _textureMatrix = new int[_patternTexture.width, _patternTexture.height];
    Color[] pixels = _patternTexture.GetPixels();
    for (int i = 0; i < _patternTexture.width; i++) 
    {
        for (int j = 0; j < _patternTexture.height; j++) 
        {
            Color currentPixel = pixels[i + (j * _patternTexture.width)];
            _textureMatrix[i, j] = currentPixel == Color.white ? 1 : 0;
        }
    }
}
\end{mypython}

La tienda (Figura \ref{fig:tienda}), por otro lado, es una interfaz de Unity basada en AnKuchen\cite{AnKuchen} (Figura \ref{fig:tiendaUML}), como todas las interfaces del juego. La tienda permite al jugador ver la información destacable de una máquina a la hora de comprarla (Restricción y valores asociados, precio y porcentaje de compleción añadido si aplica). 

\begin{figure}[H]
\centering
    \includegraphics[width=350px,clip=true]{tienda.png}
\caption{Tienda de Máquinas, compra de una máquina que restaura el 75\% de una fase}
\label{fig:tienda}
\end{figure}

\begin{figure}[H]
\centering
    \includegraphics[width=350px,clip=true]{Machine_Shop_UI.png}
\caption{Diagrama de clases de la tienda}
\label{fig:tiendaUML}
\end{figure}

Al comprar una máquina desde la tienda se crea una instancia de un PlaceableMachine y se hace una copia profunda de la máquina asociada al 'ItemHolder' de la tienda. Esto permite obtener copias lógicas de las distintas máquinas utilizando un único prefab.


\subsection{Relaciones Visuales}

Una gran parte de EcoRescue es la fase de investigación, donde el jugador debe leer y relacionar los problemas de las alteraciones con otros problemas y sus respectivas consecuencias. La implementación de esto es una serie de rutinas que instancian un 'line renderer' por cada relación, y actualizan la posición del segundo punto del 'renderer' a la posición del cursor.

\begin{mypython}[caption={Código para dibujar la línea de relaciones.},label={alg:drawloop}]
private async UniTask DrawLoop(LineRenderer line) 
{
    await UniTask.Delay(0);
    
    if (!_performDraw) 
    {
        HandleMouseUp(line, GetScreenMousePos());
        return;
    }
    Vector3 mousePos = GetMousePositionWorldPos();
    Vector3 anchorPos = GetAnchorWorldPos();
    line.positionCount = 2;
    line.SetPosition(0, anchorPos);
    line.SetPosition(1, mousePos);
    DrawLoop(line).Forget();
}
\end{mypython}

\subsection{Desarrollo Multiplataforma}

EcoRescue es un videojuego multiplataforma, desarrollado para PC y Android, el desarrollo para que sea usable en ambas plataformas no ha sido muy complejo dado que el control en PC está mayormente basado en el uso de teclado y ratón.

La mayor parte del esfuerzo ha sido sobre todo en el escalado de la UI para que sea responsive, pero gracias a la implementación de Canvas\cite{unitycanvas} de Unity, se puede configurar la interfaz para que siempre respete el margen del lado en en la que esté colocado.

Otra parte del esfuerzo ha ido en la clase de ScreenSelector, que utiliza en Unity Input System\cite{unityinputsystem} para leer los eventos de touch o click y lo utiliza para intentar obtener los objetos de tipo selectable que haya en esas coordenadas del ratón pasadas a coordenadas del mundo virtual. Si el script encuentra un selectable en esas coordenadas que está habilitado, lanza el callback de OnClick.

La única adición relevante para soportar el desarrollo multiplataforma en este caso ha sido tener que diferenciar si la acción la está realizando un touch device (Código \ref{alg:selecttouch}) o un pointer al uso (Código \ref{alg:selectPC}).

\begin{mypython}[caption={Código para seleccionar una entidad en un 'Touch Device'.},label={alg:selecttouch}]
Touch touch;
touch = Input.GetTouch(0);
selectionRay = _camera.ScreenPointToRay(touch.position);
if (Physics.Raycast(selectionRay, out hit, _range, _selectableLayers) && touch.phase == UnityEngine.TouchPhase.Began)
{
    if (!EventSystem.current.IsPointerOverGameObject(touch.fingerId))
    {
        if (_selectedObject == null || _selectedObject.gameObject != hit.collider.gameObject)
        {
            TryToSelectTouch(hit.collider.gameObject, touch);
        }
        ConfirmTouch();
    }
}
else if (touch.phase == UnityEngine.TouchPhase.Canceled)
{
    if (_selectedObject)
    {
        _selectedObject.Deselect();
        _selectedObject = null;
        onNothingSelectedCallback?.Invoke();
    }
}
\end{mypython}

\begin{mypython}[caption={Código para seleccionar una entidad en PC.},label={alg:selectPC}]
selectionRay = _camera.ScreenPointToRay(Mouse.current.position.ReadValue());
if (Physics.Raycast(selectionRay, out hit, _range))
{
    if (EventSystem.current != null)
    {
        if (!EventSystem.current.IsPointerOverGameObject())
        {
            if (_selectedObject == null || _selectedObject.gameObject != hit.collider.gameObject)
            {
                TryToSelect(hit.collider.gameObject);
            }
            CheckClick();
        }
    }
}
else
{
    if (_selectedObject)
    {
        _selectedObject.Deselect();
        _selectedObject = null;
        onNothingSelectedCallback?.Invoke();
    }
}
\end{mypython}

\subsection{Telemetría}

De cara a la validación, se ha considerado importante que, aparte de hacer un cuestionario a los alumnos de 1º de la ESO que han validado el videojuego, era necesario recabar información en relación a las partidas que jueguen. 

Con este objetivo se ha utilizado la clase DBConnection que se encarga de hacer una conexión mediante un UnityWebRequest con MariaDB cada par de minutos y hace una subida a la base de datos con información sobre el jugador y la partida (Código \ref{alg:bbdd}), de cara a obtener la mayor cantidad de datos posible.
\begin{mypython}[caption={Código para hacer inserciones en la BBDD.},label={alg:bbdd}]
public void InsertUser(int id, string username, int age, 
    string gender, int machines_placed, int machines_sold, 
    int phase_success, int phase_fail, int duration, 
    int completion, Action<int> callback)
{
    string data = \$\@"{{
                    ""username"":""TFGMVAB"", 
                    ""password"":""2024TFGjuegogestionecoPC"", 
                    ""token"":""{token}"",
                    ""table"":""users"",
                    ""data"": {
                     {""user_id"":""{id}"",
                     ""name"": ""{username}"",
                     ""age"": ""{age}"",
                     ""gender"": ""{gender}"",
                     ""machines_placed"":""{machines_placed}"",
                     ""machines_sold"" : ""{machines_sold}"",
                     ""progress"" : ""{completion}"",
                     ""success_phase"" : ""{phase_success}"",
                     ""failure_phase"" : ""{phase_fail}"",
                     ""duration"" : ""{duration}""}}
    }}";
    Debug.Log(data);
    StartCoroutine(DBAccess(data, _insert, (request) =>
    {
        if (request.result != UnityWebRequest.Result.Success)
        {
            Debug.Log(request.downloadHandler.data);
        }
        else
        {
            // La solicitud fue exitosa, puedes acceder a la respuesta
            Debug.Log(request.downloadHandler.text);
        }
        request.Dispose();
    }));
}
\end{mypython}

\subsection{Asset Importer}

Como se ha mencionado varias veces a lo largo del documento, debido al enfoque procedimental del videojuego, ha sido necesario generar un enorme volumen de datos, por desgracia, al momento de generar una cantidad similar de ScriptableObjects se pudo ver que claramente eso generaría un cuello de botella abismal en el 'workflow'. Esto provocó que se tuviera que desarrollar un 'Asset Importer' que permitiese la generación automática de los ScriptableObjects necesarios para poblar el juego con el contenido deseado.

Este 'Asset Importer' tiene dos fases:
\begin{compactitem}
    \item Generación automática de Enumerados
    \item Generación automática de ScriptableObjects
\end{compactitem}

Cada fase tiene su propio MenuItem en la barra de tareas de unity y ambas generaciones leen los archivos en formato .csv encontrados en el directorio \textunderscore Data. 

La primera fase:
\begin{enumerate}
    \item Elimina los enumerados que se encuentran en la ruta Resources
    \item Itera todos y cada uno de los archivos csv y añade todas las ocurrencias únicas de ese enumerado a un diccionario.
    \item Imprime la lista sobre un buffer de bytes.
    \item Rellena un archivo en blanco con ese buffer.
\end{enumerate} 

De esta forma generamos código programáticamente para mantener siempe actualizado el listado de enumerados (Código \ref{alg:enviroenum}).

\begin{mypython}[caption={Código para autogenerar el enumerado de EnviroConsequences.},label={alg:enviroenum}]
static string RawConsequencesEnum(string[] lines)
{
    var output = "public enum EnviroConsequenceType{";
    foreach (var consequence in lines)
    {
        if (string.IsNullOrEmpty(consequence))
        {
            continue;
        }
        var tags = consequence.Split("|");
        output += GetTypeTagName(tags[3].Trim()) + ", ";
    }
    output += "}";
    return output;
}
\end{mypython}

La segunda fase:
\begin{enumerate}
    \item Itera por cada linea de cada archivo .csv.
    \item Genera una instancia de tipo 'Type' DTO (Data Transfer Object).
    \item Añade esa instancia a un diccionario que relaciona cada enumerado con su respectivo DTO (Código \ref{alg:enviroSO1}). 
    \item Una vez iteradas todas las líneas, se iteran todos las claves del diccionario.
    \item Por cada clave se crea un ScriptableObject de su respectivo tipo, y se hace una copia profunda de datos desde el DTO (Código \ref{alg:enviroSO2}) al ScriptableObject recién generado.
\end{enumerate} 

\begin{mypython}[caption={Código para autogenerar DTOs de tipo EnviroConsequences.},label={alg:enviroSO1}]
static Dictionary<EnviroConsequenceType, EnviroConsequence.EnviroConsequenceDTO> ParseConsequenceDict()
{
    var dict = new Dictionary<EnviroConsequenceType, EnviroConsequence.EnviroConsequenceDTO>();
    var lines = ConsequenceString();
    foreach (var line in lines)
    {
        if (string.IsNullOrEmpty(line))
            continue;
        var tags = line.Split("|");
        var dto = new EnviroConsequence.EnviroConsequenceDTO();
        dto.name = tags[0].Trim();
        dto.Title = tags[1].Trim();
        dto.Description = tags[2].Trim();
        dto.Type = (EnviroConsequenceType) Enum.Parse(typeof(EnviroConsequenceType), tags[3].Trim());
        dto.Sprite = FindSprite(tags[4].Trim());
        dto.color = GetColor(tags[5].Trim());
        dict[dto.Type] = dto;
    }
    return dict;
}
\end{mypython}

\begin{mypython}[caption={Código para autogenerar ScriptableObject de tipo EnviroConsequences.},label={alg:enviroSO2}]
static Dictionary<EnviroConsequenceType, EnviroConsequence> CreateConsequences()
{
    var consequences = new Dictionary<EnviroConsequenceType, EnviroConsequence>();
    var consequenceArray = Resources.LoadAll("ScriptableObjects/Consequences", typeof(EnviroConsequence));
    foreach (var consequence in consequenceArray.Cast<EnviroConsequence>())
    {
        consequences[consequence.Type] = consequence;
    }
    var consequenceDTODict = ParseConsequenceDict();
    foreach (var type in (EnviroConsequenceType[])System.Enum.GetValues(typeof(EnviroConsequenceType)))
    {
        if (!consequences.ContainsKey(type))
        {
            var newConsequence = ScriptableObject.CreateInstance<EnviroConsequence>();
            string assetPath = ImportMarroneroPath + "ScriptableObjects/" + "Consequences" + "/" + type.ToString() + ".asset";
            newConsequence.name = consequenceDTODict[type].name;
            newConsequence.Title = consequenceDTODict[type].Title;
            newConsequence.Description = consequenceDTODict[type].Description;
            newConsequence.Type = consequenceDTODict[type].Type;
            newConsequence.Sprite = consequenceDTODict[type].Sprite;
            newConsequence.color = consequenceDTODict[type].color;
            AssetDatabase.CreateAsset(newConsequence, assetPath);
            EditorUtility.SetDirty(newConsequence);
            AssetDatabase.SaveAssets();
            AssetDatabase.Refresh();
            EditorUtility.FocusProjectWindow();
            Selection.activeObject = newConsequence;
            consequences[type] = newConsequence;
        }
    }
    return consequences;
}
\end{mypython}

\subsection{AutoSlicer}

Se ha desarrollado también una pequeña herramienta usando el plugin Auto9Slicer\cite{Auto9Slicer} de kyubuns que permite hacer 'slice' automáticamente de sprites en formato png (Código \ref{alg:spriteslicer}), de forma que se puedan estirar sin deformarse. 

\begin{mypython}[caption={Código para aplicar automáticamente el algoritmo de slice sobre los sprites deseados.},label={alg:spriteslicer}]
[CustomEditor(typeof(Auto9SliceTester))]
public class Auto9SliceTesterEditor : Editor
{
    public override void OnInspectorGUI()
    {
        base.OnInspectorGUI();
        EditorGUILayout.Space(20);
        if (GUILayout.Button("Run")) ((Auto9SliceTester) target).Run();
    }
}

[CreateAssetMenu(menuName = "Auto 9Slice/Tester", fileName = nameof(Auto9SliceTester))]
public class Auto9SliceTester : ScriptableObject
{
    public SliceOptions Options => options;
    [SerializeField] private SliceOptions options = new SliceOptions();
    public bool CreateBackup => createBackup;
    [SerializeField] private bool createBackup = true;
    public void Run()
    {
        var directoryPath = Path.GetDirectoryName(AssetDatabase.GetAssetPath(this));
        if (directoryPath == null) throw new Exception(\$"directoryPath == null");
        var fullDirectoryPath = Path.Combine(Path.GetDirectoryName(Application.dataPath) ?? "", directoryPath);
        var targets = Directory.GetFiles(fullDirectoryPath)
            .Select(Path.GetFileName)
            .Where(x => x.EndsWith(".png") || x.EndsWith(".jpg") || x.EndsWith(".jpeg"))
            .Where(x => !x.Contains(".original"))
            .Select(x => Path.Combine(directoryPath, x))
            .Select(x => (Path: x, Texture: AssetDatabase.LoadAssetAtPath<Texture2D>(x)))
            .Where(x => x.Item2 != null)
            .ToArray();
        foreach (var target in targets)
        {
            var importer = AssetImporter.GetAtPath(target.Path);
            if (importer is TextureImporter textureImporter)
            {
                if (textureImporter.spriteBorder != Vector4.zero) continue;
                var fullPath = Path.Combine(Path.GetDirectoryName(Application.dataPath) ?? "", target.Path);
                var bytes = File.ReadAllBytes(fullPath);
                if (CreateBackup)
                {
                    var fileName = Path.GetFileNameWithoutExtension(fullPath);
                    File.WriteAllBytes(Path.Combine(Path.GetDirectoryName(fullPath) ?? "", fileName + ".original" + Path.GetExtension(fullPath)), bytes);
                }
                var targetTexture = new Texture2D(2, 2);
                targetTexture.LoadImage(bytes);
                var slicedTexture = Slicer.Slice(targetTexture, Options);
                textureImporter.textureType = TextureImporterType.Sprite;
                textureImporter.spriteBorder = slicedTexture.Border.ToVector4();
                if (fullPath.EndsWith(".png")) File.WriteAllBytes(fullPath, slicedTexture.Texture.EncodeToPNG());
                if (fullPath.EndsWith(".jpg")) File.WriteAllBytes(fullPath, slicedTexture.Texture.EncodeToJPG());
                if (fullPath.EndsWith(".jpeg")) File.WriteAllBytes(fullPath, slicedTexture.Texture.EncodeToJPG());
                Debug.Log(\$"Auto 9Slice {Path.GetFileName(target.Path)} = {textureImporter.spriteBorder}");
            }
        }
        AssetDatabase.Refresh();
    }
}
\end{mypython}

% Capítulo 4
\chapter{Validación}
\label{sec:validacion}

\section{Encuestas}

De cara a validar el prototipo, la primera fuente de datos planteada fue preparar una encuesta (Disponible en el Anexo \ref{sec:apendice}) con la que obtener sensaciones e impresiones de primera mano de los alumnos.

El objetivo de estas preguntas es tanto obtener una idea general de si el prototipo verdaderamente está sirviendo como ayuda para desarrollar el PC, como comprobar si está sirviendo para divulgar acerca de la restauración de cosistemas, y además comprobar cómo de divertido y usable es como videojuego.

El resultado hn sido muy útil e ilustrador, dado que el feedback ha sido increíblemente positivo. 

\section{Datos recabados}

Por desgracia los datos recabados son ligeramente dispares y no representan la muestra total de alumnos que evaluamos con el juego (~20 alumnos), ya que el instituto tenía una serie de Firewalls que acabaron bloqueando las conexiones del prototipo con la BBDD, no obstante, los datos que sí hay pueden usarse.

Los datos recabados finalmente son:

\begin{compactitem}
    \item User\textunderscore Id - El id inequívoco de un usuario.
    \item Name - el usuario que hayan elegido.
    \item Gender - el Icono que hayan elegido.
    \item Age - la edad del usuario.
    \item Progress - la cantidad media de progreso de todas las fases (máximo 100).
    \item Machines Placed - La cantidad de máquinas colocadas.
    \item Machines Sold - La cantidad de máquianas vendidas.
    \item Success Phase - La cantidad de fases completadas.
    \item Failure Phase - La cantidad de fases reiniciadas.
    \item Duration - El tiempo de juego.
\end{compactitem}

\section{Experiencia de Validación}

La experiencia de la validación fue en general muy positiva, se pudo observar a los niños interactuar en tiempo real con el juego y ver dónde les costaba avanzar. El resultado es una vez más muy positivo dado que todos los alumnos menos dos fueron capaces de completar la demo. No obstante, la gran mayoría se atascó en la fase de relacionar problemas con otros problemas y consecuencias, por lo que quizás ese apartado no está bien explicado o es demasiado complejo para niños de esas edades.

Por otra parte, el ambiente en la clase fue muy bueno y la experiencia fue positiva y amena para todos los implicados, los profesores estuvieron de acuerdo en que utilizar este tipo de herramientas de forma didáctica es una muy buena forma de mejorar el ambiente en el aula sin dejar de enseñar a los alumnos.

\section{Resultados}

En primer lugar, podemos ver que en la representación de género (Figura \ref{fig:questionario_2}) ha habido bastante paridad (58 - 42\%). Además de poder encontrar que la distribución de regularidad a la hora de jugara videojuegos también es bastante equitativa (Figura \ref{fig:questionario_3}) donde se puede observar que más o menos la mitad de los alumnos jugaban a diario, y el resto se dividían entre semanalmente y casi nunca.

Además, en las Figuras \ref{fig:questionario_4}, \ref{fig:questionario_5}, \ref{fig:questionario_6} y \ref{fig:questionario_7} se puede observar cómo los alumnos sí que han entrenado su pensamiento computacional, donde el 67\% afirma haber practicado la depuración de errores, el 87\% comenta haber seguido un orden específico a la hora de seleccionar restauraciones que practicar (Lo que implica que estaban desarrollando algoritmia) y, finalmente, el 92\% de alumnos ha desarrollado el Análisis de Datos y la Generalización al leer los datos de las alteraciones y probolemas, y utilizar dicha información a la hora de restaurar problemas similares de distintos biomas.

Estas conclusiones se ven reflejadas en los datos recogidos en la Base de Datos, donde se puede ver claramente que pese a que de media los alumnos tuvieron que reiniciar las fases 3 o 4 veces, la gran mayoría consiguieron completar toda la demo, también se puede observar un alto grado de experimentación a la hora de jugar, dado que prácticamente todos los alumnos vendieron 3 o más máquinas. 

Es relevante quizás mencionar que 'Pako' y 'Paula' no fueron capaces de completar el juego, por los resultados podemos observar que 'Pako' se quedó atascado reintentando la misma fase una y otra vez, probablemente intentando colocar una máquina con 90\% de probabilidad de fallo, y 'Paula' no vendió ni una sola máquina, quizás provocando que se quedase sin dinero.

Otros datos a tener en cuenta son los de la Tabla \ref{fig:tablaResultadosPC}, donde se pueden encontrar testimonios en los que los alumnos indican sus partes favoritas del prototipo. En general queda patente que el sentimiento general hacia el juego es positivo, la opinión es que el juego es 'divertido', 'original' y 'fácil de usar', pero además, merece la pena destacar que los propios alumnos opinan que ha sido una experiencia de aprendizaje amena y divertida, 'mucho más divertida' al ser 'comparada con otros utensilios de clase'.

% Capítulo 5

\chapter{Conclusiones y Trabajos Futuros}
\label{sec:conclusiones}

\input{pages/conclusiones.tex}

\blankpage


%%%%%%%%%%%%%%%%%%%%%%%%%%%%%%% Bibliografía %%%%%%%%%%%%%%%%%%%%%%%%%%%%%%%

\phantomsection
\addcontentsline{toc}{chapter}{Bibliografía}

\footnotesize{
%\bibliographystyle{hispa}
\bibliographystyle{IEEEtran}
\bibliography{bibliografia}
}



% No expandir elementos para llenar toda la página
\raggedbottom
\newpage

%%%%%%%%%%%%%%%%%%%%%%%%%%%%%%% Apéndices %%%%%%%%%%%%%%%%%%%%%%%%%%%%%%%

\appendix

\phantomsection
\addcontentsline{toc}{chapter}{Anexos}

\mbox{}
\vfill
\begin{center}
\begin{Huge}
\textbf{Apéndices}
\end{Huge}
\end{center}
\vfill
\mbox{}
\thispagestyle{empty}

\mbox{}
\thispagestyle{empty}


% Primer apéndice
\chapter{Anexo}
\label{sec:apendice}

\section{Documentos y Ejecutables}
\subsection{Game Design Document}

\href{https://github.com/dexaxi/RecursosSolarcore/blob/main/Assets/Documentation/Memoria%20-%20%C3%81ngel/tfg-template-master/EcoRescue.pdf}{Enlace al GDD}

\subsection{Repositorios}
\begin{itemize}
  \item EcoRescue: \url{https://github.com/dexaxi/RecursosSolarcore}
  \item DUJAL: \url{https://github.com/dexaxi/TFG\textunderscore Unity\textunderscore Package}
  \item AkyuiUnity: \url{https://github.com/kyubuns/AkyuiUnity/tree/main} 
  \item AnKuchen: \url{https://github.com/kyubuns/AnKuchen}
  \item Auto9Slicer: \url{https://github.com/kyubuns/Auto9Slicer}
  \item UniTask: \url{https://github.com/Cysharp/UniTask}
\end{itemize}

\subsection{Redes de EcoRescue}
Descarga el juego: \url{https://dexaxi.itch.io/eco-rescue}

Disponible en Windows, Linux, Mac, Android y WebGL

\section{Tablas y figuras}

\subsection{Diagrama de Clases completo}

\begin{figure}[H]
  \centering
  \includegraphics[width=450px,clip=true]{Logic_Class_Diagram.png}
  \caption{Diagrama de Clases completo}
  \label{fig:logicUML}
\end{figure}
\raggedbottom


\subsection{Resultados del Cuestionario}
% Insertar una figura
\begin{figure}[H]
  \centering
  \includegraphics[width=450px,clip=true]{questionario_1.png}
  \caption{¿Cuántos años tienes?}
  \label{fig:questionario_1}
\end{figure}
\raggedbottom

\begin{figure}[H]
  \centering
  \includegraphics[width=450px,clip=true]{questionario_2.png}
  \caption{¿Cómo te identificas?}
  \label{fig:questionario_2}
\end{figure}
\raggedbottom

\begin{figure}[H]
  \centering
  \includegraphics[width=450px,clip=true]{questionario_3.png}
  \caption{¿Con qué frecuencia juegas a videojuegos?}
  \label{fig:questionario_3}
\end{figure}
\raggedbottom

\begin{figure}[H]
  \centering
  \includegraphics[width=450px,clip=true]{questionario_4.png}
  \caption{Mientras jugabas a un nivel, ¿ha habido algún momento en el que hayas cometido un error y hayas tenido que encontrar y arreglar el problema?}
  \label{fig:questionario_4}
\end{figure}
\raggedbottom

\begin{figure}[H]
  \centering
  \includegraphics[width=450px,clip=true]{questionario_5.png}
  \caption{A la hora de restaurar un ecosistema, ¿Ha sido importante seguir un orden específico de tareas?}
  \label{fig:questionario_5}
\end{figure}
\raggedbottom

\begin{figure}[H]
  \centering
  \includegraphics[width=450px,clip=true]{questionario_6.png}
  \caption{Al principio del nivel, ¿la información que te dan las máquinas sobre los ecosistemas te ha ayudado a saber cómo arreglarlos?}
  \label{fig:questionario_6}
\end{figure}
\raggedbottom

\begin{figure}[H]
  \centering
  \includegraphics[width=450px,clip=true]{questionario_7.png}
  \caption{A la hora de resolver un problema, ¿te resulta útil lo que has aprendido en los biomas anteriores?}
  \label{fig:questionario_7}
\end{figure}
\raggedbottom

\begin{figure}[H]
  \centering
  \includegraphics[width=450px,clip=true]{questionario_8.png}
  \caption{¿Te ha gustado el juego?}
  \label{fig:questionario_8}
\end{figure}
\raggedbottom

\begin{figure}[H]
  \centering
  \includegraphics[width=450px,clip=true]{questionario_9.png}
  \caption{¿Cómo de divertido te ha parecido el juego?}
  \label{fig:questionario_9}
\end{figure}
\raggedbottom

\begin{figure}[H]
  \centering
  \includegraphics[width=450px,clip=true]{questionario_10.png}
  \caption{¿Cómo crees que es la dificultad del juego?}
  \label{fig:questionario_10}
\end{figure}
\raggedbottom

\begin{figure}[H]
  \centering
  \includegraphics[width=450px,clip=true]{questionario_11.png}
  \caption{¿Cómo de sencillo te ha parecido utilizar los controles del juego?}
  \label{fig:questionario_11}
\end{figure}
\raggedbottom

\begin{figure}[H]
  \centering
  \includegraphics[width=450px,clip=true]{questionario_12.png}
  \caption{¿Has aprendido algo nuevo sobre la restauración de ecosistemas mientras jugabas?}
  \label{fig:questionario_12}
\end{figure}
\raggedbottom

\begin{table}[H]
  \begin{center}
  \setlength{\tabcolsep}{5pt}
  \renewcommand{\arraystretch}{1.2}
  \begin{tabular}{ | m{\textwidth} | } 
    \hline
    'La información que dan es muy útil' \\ 
    'He podido aprender mientras jugaba' \\ 
    'Me han gustado mucho los diseños, también me ha gustado cuando ponías las máquinas para ayudar al ecosistema' \\ 
    'La originalidad y poder aprender más sobre los ecosistemas' \\ 
    'Es fácil de usar y te explica muy bien las cosas' \\ 
    'Me ha gustado mucho la decoración y la base del juego, creo que es mucho más divertido comparado con los otros utensilios de clase' \\ 
    'Me gusta como iba subiendo la barra de progreso' \\ 
    'Las diferentes funciones de las máquinas' \\ 
    \hline
  \end{tabular}
  \centering
  \caption{¿Qué es lo que más te ha gustado del juego?}
  \label{fig:tablaResultadosPC}
  \end{center}
\end{table}  
 
\subsection{Datos Obtenidos por Telemetría}
\begin{table}[H]
  \begin{center}
  \setlength{\tabcolsep}{5pt}
  \renewcommand{\arraystretch}{1.2}
  \hspace*{-40px}
  \begin{tabular}{ @{} | m{2em} | m{4em} | m{3em} | m{2em} | m{4em} | m{4em} | m{4em} | m{3em} | m{3em} | m{4em} |  } 
  \hline
  ID & Name & Gender & Age & Progress & Machines Placed  & Machines Sold & Success Phase & Failure Phase & Duration \\
  \hline
  121  & 'Sandia' & 'Chica' &12 &0 &0 &0 &0 &0 &0  \\
  \hline
  121  & 'Sandia' & 'Chica' &12 &11 &2 &1 &1 &0 &2  \\
  \hline
  121  & 'Sandia' & 'Chica' &12 &22 &6 &4 &2 &3 &4  \\
  \hline
  121  & 'Sandia' & 'Chica' &12 &55 &9 &4 &5 &5 &6  \\
  \hline
  121  & 'Sandia' & 'Chica' &12 &100 &14 &5 &9 &9 &8  \\
  \hline
  122  & 'Ana' & 'Chica' &12 &0 &0 &0 &0 &0 &0  \\
  \hline
  122  & 'Ana' & 'Chica' &12 &22 &5 &3 &2 &2 &2  \\
  \hline
  122  & 'Ana' & 'Chica' &12 &44 &7 &3 &4 &4 &4  \\
  \hline
  122  & 'Ana' & 'Chica' &12 &100 &14 &5 &9 &9 &6  \\
  \hline
  123  & 'Ireee' & 'Chica' &13 &0 &0 &0 &0 &0 &0  \\
  \hline
  123  & 'Ireee' & 'Chica' &13 &33 &5 &2 &3 &2 &2  \\
  \hline
  123  & 'Ireee' & 'Chica' &13 &55 &8 &3 &5 &3 &4  \\
  \hline
  123  & 'Ireee' & 'Chica' &13 &100 &12 &3 &9 &5 &6  \\
  \hline
  124  & 'Bea' & 'Chica' &12 &0 &0 &0 &0 &0 &0  \\
  \hline
  124  & 'Bea' & 'Chica' &12 &33 &7 &4 &3 &2 &2  \\
  \hline
  124  & 'Bea' & 'Chica' &12 &44 &9 &5 &4 &3 &4  \\
  \hline
  124  & 'Bea' & 'Chica' &12 &100 &14 &5 &9 &5 &6  \\
  \hline
  125  & 'Leo' & 'Chico' &12 &0 &0 &0 &0 &0 &0  \\
  \hline
  125  & 'Leo' & 'Chico' &12 &22 &3 &1 &2 &0 &2  \\
  \hline
  125  & 'Leo' & 'Chico' &12 &33 &5 &2 &3 &3 &4  \\
  \hline
  125  & 'Leo' & 'Chico' &12 &55 &7 &2 &5 &4 &6  \\
  \hline
  126  & 'SALVA' & 'Chico' &13 &0 &0 &0 &0 &0 &0  \\
  \hline
  126  & 'SALVA' & 'Chico' &13 &33 &6 &3 &3 &2 &2  \\
  \hline
  126  & 'SALVA' & 'Chico' &13 &55 &8 &3 &5 &3 &4  \\
  \hline
  126  & 'SALVA' & 'Chico' &13 &100 &12 &3 &9 &5 &6  \\
  \hline
  127  & 'Calvo Pro' & 'Chico' &12 &0 &0 &0 &0 &0 &0 \\
  \hline
  127  & 'Calvo Pro' & 'Chico' &12 &22 &7 &5 &2 &1 &2 \\
  \hline 
  127  & 'Calvo Pro' & 'Chico' &12 &33 &8 &5 &3 &2 &4 \\
  \hline 
\end{tabular}
\centering
\end{center}
\end{table} 

\begin{table}[H]
  \begin{center}
  \setlength{\tabcolsep}{5pt}
  \renewcommand{\arraystretch}{1.2}
  \hspace*{-67px}
  \begin{tabular}{ @{} | m{2em} | m{4em} | m{3em} | m{2em} | m{4em} | m{4em} | m{4em} | m{3em} | m{3em} | m{4em} |  } 
  \hline
  128  & 'Virgy' & 'Chica' &12 &0 &0 &0 &0 &0 &0  \\
  \hline
  128  & 'Virgy' & 'Chica' &12 &33 &7 &4 &3 &2 &2  \\
  \hline
  128  & 'Virgy' & 'Chica' &12 &55 &9 &4 &5 &4 &4  \\
  \hline
  128  & 'Virgy' & 'Chica' &12 &100 &13 &4 &9 &5 &6  \\
  \hline
  129  & 'Sara' & 'Chica' &12 &0 &0 &0 &0 &0 &0  \\
  \hline
  129  & 'Sara' & 'Chica' &12 &33 &4 &1 &3 &2 &2  \\
  \hline
  129  & 'Sara' & 'Chica' &12 &44 &5 &1 &4 &3 &4  \\
  \hline
  129  & 'Sara' & 'Chica' &12 &100 &10 &1 &9 &7 &6  \\
  \hline
  130  & 'Paula' & 'Chica' &12 &0 &0 &0 &0 &0 &0  \\
  \hline
  130  & 'Paula' & 'Chica' &12 &11 &1 &0 &1 &1 &2  \\
  \hline
  130  & 'Paula' & 'Chica' &12 &33 &3 &0 &3 &2 &4  \\
  \hline
  130  & 'Paula' & 'Chica' &12 &77 &7 &0 &7 &6 &6  \\
  \hline
  131  & 'Pako' & 'Chico' &13 &0 &0 &0 &0 &0 &0  \\
  \hline
  132  & 'Pako' & 'Chico' &13 &22 &3 &1 &2 &1 &2  \\
  \hline
  133  & 'Pako' & 'Chico' &13 &55 &7 &2 &5 &2 &4  \\
  \hline
  134  & 'Pako' & 'Chico' &13 &66 &10 &4 &6 &5 &6  \\
  \hline
  135  & 'Pako' & 'Chico' &13 &66 &10 &4 &6 &6 &8  \\
  \hline
  136  & 'Pako' & 'Chico' &13 &66 &10 &4 &6 &7 &10  \\
  \hline
  137  & 'Pako' & 'Chico' &13 &66 &10 &4 &6 &8 &12  \\
  \hline
  138  & 'Pako' & 'Chico' &13 &66 &10 &4 &6 &9 &14  \\
  \hline
  139  & 'Pako' & 'Chico' &13 &66 &10 &4 &6 &11 &16  \\
  \hline
  140  & 'Pako' & 'Chico' &13 &66 &10 &4 &6 &12 &18  \\
  \hline
  141  & 'Pako' & 'Chico' &13 &66 &10 &4 &6 &12 &20  \\
  \hline
  147  & 'Paco' & 'Chico' &13 &0 &0 &0 &0 &0 &0  \\
  \hline
  147  & 'Paco' & 'Chico' &13 &11 &2 &1 &1 &0 &2  \\
  \hline
  147  & 'Paco' & 'Chico' &13 &33 &4 &1 &3 &1 &4  \\
  \hline
  147  & 'Paco' & 'Chico' &13 &66 &7 &1 &6 &5 &6  \\
  \hline
  147  & 'Paco' & 'Chico' &13 &100 &10 &1 &9 &8 &8  \\
  \hline
  148  & 'cali' & 'Chico' &13 &0 &0 &0 &0 &0 &0  \\
  \hline
  148  & 'cali' & 'Chico' &13 &22 &5 &3 &2 &2 &2  \\
  \hline
  148  & 'cali' & 'Chico' &13 &44 &7 &3 &4 &4 &4  \\
  \hline
  148  & 'cali' & 'Chico' &13 &100 &13 &4 &9 &5 &6  \\
  \hline
\end{tabular}
\centering
\end{center}
\end{table}  

\begin{table}[H]
  \begin{center}
  \hspace*{-40px}
  \begin{tabular}{ @{} | m{2em} | m{4em} | m{3em} | m{2em} | m{4em} | m{4em} | m{4em} | m{3em} | m{3em} | m{4em} |  } 
  \hline
  149  & 'Joselito' & 'Chico' &13 &0 &0 &0 &0 &0 &0  \\
  \hline
  150  & 'Joselito' & 'Chico' &13 &11 &2 &1 &1 &0 &2  \\
  \hline
  151  & 'Joselito' & 'Chico' &13 &22 &7 &5 &2 &0 &4  \\
  \hline
  152  & 'Joselito' & 'Chico' &13 &33 &8 &5 &3 &1 &6  \\
  \hline
  153  & 'Joselito' & 'Chico' &13 &55 &11 &6 &5 &1 &8  \\
  \hline
  154  & 'Joselito' & 'Chico' &13 &100 &15 &6 &9 &3 &10  \\
  \hline
\end{tabular}
\centering
\caption{Resultados obtenidos por el Prototipo}
\label{fig:tablabbddpc}
\end{center}
\end{table}  


% También con \pagebreak

% Fin del documento
\end{document}
