Descripción de los lenguajes de programación, entornos de desarrollo, herramientas auxiliares, librerías de terceros, sistemas operativos, navegadores web, etc… utilizados para la realización del proyecto así como la metodología empleada. 
El grado de profundidad a la hora de explicar cada tecnología dependerá de lo relevante que ha sido para el proyecto y lo conocida que es. 
Por ejemplo, si se usa el lenguaje de programación Java, no es necesario entrar en tanto detalle que si se usa un lenguaje mucho menos usado como Scala, por ejemplo. Respecto a la metodología, dada la naturaleza de los proyectos, se suele describir una metodología iterativa e incremental en espiral, en la que se van sucediendo reuniones con el profesor que van definiendo el ámbito del proyecto. Este capítulo puede tener una extensión entre 10 y 15 páginas.
\section{Herramientas y Tecnologías}

\subsection{C\# y Unity}

Se ha utilizado Unity\cite{unity} como motor para el desarrollo de EcoRescue y C\#\cite{csharp} como lenguaje de programación, se ha elegido este motor dado que era el motor público con el que los desarrolladores teníamos más experiencia.

\subsection{Git \& Github}

Github \cite{github} es un entorno de desarrollo y control de versiones colaborativo basado en la tecnología Git 

\section{Metodologías}

Aquí se detallarán las metodologías de desarrollo de software utilizadas (usualmente iterativa e incremental). 
También se puede hablar de metodologías de gestión de proyectos como Scrum, Kanban, etc... 
Aquí también se hablará del modelo de desarrollo utilizado con Git (GitFlow, GitHub Flow, etc...)
